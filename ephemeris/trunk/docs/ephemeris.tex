\input texinfo   @c -*-texinfo-*-
@comment $Id$
@comment %**start of header
@setfilename ephemeris.info

@finalout
@c @setchapternewpage odd
@firstparagraphindent none
@syncodeindex pg cp

@iftex
@afourpaper
@end iftex

@include version.texi
@set VERSION @value{MAJOR_VERSION}.@value{MINOR_VERSION}

@settitle Ada Ephemeris Library @value{VERSION}

@comment %**end of header
@c ======================================================================
@copying
This manual is for the Ada Ephemeris Library (version @value{VERSION},
@value{UPDATED}).

$Id$

Copyright @copyright{} 2006 Juan A. de la Puente

@quotation
Permission is granted to copy, distribute and/or modify this document
under the terms of the GNU Free Documentation License, Version 1.1 or
any later version published by the Free Software Foundation; with no
Invariant Sections, no Front-Cover Texts, and no Back-Cover Texts.  A
copy of the license is included in the section entitled ``GNU Free
Documentation License.''
@end quotation


@end copying
@c ======================================================================
@titlepage

@title Ada Ephemeris Library
@subtitle Version @value{VERSION}
@subtitle Date: @today{}
@author Juan A. de la Puente

@page
@vskip 0pt plus 1filll
@insertcopying
@end titlepage

@contents
@c ======================================================================
@ifnottex
@node Top, Introduction, (dir), (dir)
@top Ada Ephemeris Library

@insertcopying

The Ada Ephemeris Library is a collection of Ada packages that
provide access to the Solar System Ephemeris developed by the
Jet Propulsion Laboratory (JPL).
@end ifnottex

@menu
* Introduction::                Purpose and related information.
* Overview::                    
* User's Guide::                
* Detailed design::             
* Ephemeris file formats::      
* Building the library on Windows systems::  
* Copying this manual::         License information
* Index::                       
@end menu

@c ======================================================================

@node Introduction, Overview, Top, Top
@chapter Introduction

@menu
* Purpose::                     
* Installation requisites::     
* Acknowledgments::             
@end menu

@node Purpose, Installation requisites, Introduction, Introduction
@section Purpose

This manual contains instructions for installing and using the Ada
Ephemeris Library (@acronym{AEL}), a collection of Ada packages
that provide access to the Solar System Ephemerides developed by the
Jet Propulsion Laboratory (@acronym{JPL})@footnote{The @acronym{JPL}
has developed several kinds of ephemerides (e.g.  @acronym{DE200},
@acronym{DE405}, and @acronym{DE406}).  The current version of
@acronym{AEL} has been designed to work with different ephemeris
formats, but has only been tested with the @acronym{DE200} series.}.

The @acronym{AEL} distribution consists of a set of packages defining
constants and library functions that can be used in other Ada programs
to read ephemeris data, and some utility programs which are needed to
install and test the ephemeris data files. The data files themselves
are not part of the @acronym{AEL} distribution, and should be
downloaded by the user directly from the @acronym{JPL} site@footnote{
@uref{http://ssd.jpl.nasa.gov}}.

Previous knowledge of the Ada programming language is
assumed. The reader should also be familiar with
@acronym{GNAT}@footnote{ @acronym{GNAT,@acronym{GNU}-@acronym{NYU} Ada
Translator} is a free programming environment for Ada.  See the AdaCore web site
@uref{http://libre.adacore.com} for more details.} or some other
programming environment for Ada 2005.

@c ----------------------------------------------------------------------
@node    Installation requisites, Acknowledgments, Purpose, Introduction
@section Installation requisites

The @acronym{AEL} software is highly portable and can be used on any
platform for which an Ada 2005 compilation system is available. The
distributed version has been compiled with @acronym{GNAT GPL} 2010 on 
PC workstations running @acronym{GNU}/Linux and on Mac computers 
running MacOS X Snow Leopard. Compiling the software with other Ada 2005 
compilers should no give any problems as only Ada 2005 standard language 
features have been used.

@c ----------------------------------------------------------------------
@node Acknowledgments,  , Installation requisites, Introduction
@section Acknowledgments

The information on the @acronym{JPL} ephemerides that has been used to
develop the Ada Ephemeris Library comes from the following sources:

@itemize @bullet
@item The @acronym{JPL} ephemerides package user's guide@footnote{
@uref{ftp://ssd.jpl.nasa.gov/pub/eph/planets/usrguide}}.

@item David Hoffman's user's guide for the C utility programs for
processing @acronym{JPL} Ephemeris Data@footnote{
@uref{ftp://ssd.jpl.nasa.gov/pub/eph/planets/C-versions/hoffman}}.

@item The @emph{Explanatory Supplement to the Astronomical Almanac},
edited by P. Kenneth Seidelmann, University Science Books, 1992,
especially chapter 5.
@end itemize

@c ======================================================================
@node Overview, User's Guide, Introduction, Top
@chapter Overview

@menu
* Fundamental ephemerides::     
* The JPL ephemeris files::     
* The Ada Ephemeris Library::   
@end menu


@c ......................................................................
@node Fundamental ephemerides, The JPL ephemeris files, Overview, Overview
@section Fundamental ephemerides

The fundamental ephemerides of the solar system bodies are the
computations of the body states of the bodies, i.e their positions and
velocities with respect to some reference coordinate frame.  The
motion of the solar system bodies is governed by Newton's laws
corrected for general and special relativity effects. Integration of
these equations fitting observational data is the basis for the
computation of modern ephemerides data. The resulting values are
stored in computer files in order to make them available for
astronomical computations.

The computation of position and velocity values at every instant of
time is clearly not feasible. The solution is to use an interpolation
method in order to get a compact representation of the state
values. The @acronym{JPL} ephemerides use Chebyshev polynomial
interpolation in such a way that the state of a body at any time
within the ephemeris range can be efficiently computed from a small
number of coefficients. The number of coefficients and the precision
of the data are adjusted so that the interpolation error is less than
0.5mm. @xref{Interpolation coefficients}, for the details.

The position and velocity components are computed in barycentric
rectangular equatorial coordinates, referred to the mean equator and
equinox of J2000.0. In this reference system the origin of coordinates
is the center of mass of the solar system, the X-Y plane is parallel
to the celestial equator plane, i.e. the plane that is normal to the
axis of rotation of the Earth, and the X axis has the direction of the
equinox, i.e. the intersection of the equator and the ecliptic planes.
The units are km and km/day for the position and velocity,
respectively.  The time independent variable is @acronym{TDB,
barycentric dynamical time} in Julian days.  See the @cite{Explanatory
Supplement to the Astronomical Almanac} for the details.

Since the orientation of the Earth's equator and the direction of the
equinox are changing, the reference system is defined with respect to
a particular epoch.  The reference epoch for the @acronym{DE200}
ephemeris is J2000.0, which is defined as Julian date 2451545.0
(January 1st, 2000, at 12:00).

@c ......................................................................
@node The JPL ephemeris files, The Ada Ephemeris Library, Fundamental ephemerides, Overview
@section The @acronym{JPL} ephemeris files

The @acronym{JPL} ephemeris files contain the Chebyshev coefficients
for the position and velocity of the main solar system bodies across a
given time span. The coefficients are grouped into records, each of
them containing data for a 32-day long @emph{interval}. Intervals are
further subdivided into @emph{granules}, with a length that depends on
the body under consideration. The coefficients enable the position and
velocity components to be computed for each major solar system
body. @ref{Ephemeris file formats}, contains a detailed description of
the @acronym{JPL} source files and the binary files used by
@acronym{AEL}.

@c ......................................................................
@node The Ada Ephemeris Library,  , The JPL ephemeris files, Overview
@section The Ada Ephemeris Library

The Ada Ephemeris Library is a set of Ada packages that provide
high-level access to the state of the main solar system bodies:
Mercury, Venus, the Earth, Mars, Jupiter, Saturn, Uranus, Neptune,
Pluto, the Moon, and the Sun@footnote{The ephemeris files contain data
for a slightly different set of bodies: Mercury, Venus, the Earth/Moon
barycenter, Mars, Jupiter, Saturn, Uranus, Neptune, Pluto, the Moon
(in geocentric coordinates), and the Sun, plus nutation angles and, in
some cases, libration angles. The transformations for the Earth and
the Moon are carried out internally in the @acronym{AEL} operations,
and the nutation angles are not being used in this version of
@acronym{AEL}.}. The position and velocity values are given in
@acronym{AU}@footnote{An @acronym{AU, Astronomical Unit} is a unit of
distance which equals @math{1.4959787066 \times 10^{11}} m.} and
@acronym{AU}/day@footnote{ The astronomical day is defined as 86400
s.}. Notice that the coefficients in the ephemeris files give the
state components in km and km/day, respectively. The unit
transformations are carried out internally in the @acronym{AEL}
operations.

The main packages in the @acronym{AEL} are @code{Ephemeris} and
@code{Ephemeris.Data}. The former is the root package of the library,
in which the supported ephemeris sets (@acronym{DE200},
@acronym{DE405}, and @acronym{DE406})@footnote{Notice that the current
version has only been tested with the @acronym{DE200} ephemeris.} are
defined.  The @code{Ephemeris.Data} package contains all the necessary
definitions to get the state of a body at a given time. The next
chapter (@pxref{User's Guide}) contains a detailed description of
these two packages.

@c ======================================================================
@node  User's Guide, Detailed design, Overview, Top
@chapter User's Guide 

@c ----------------------------------------------------------------------
@menu
* Installing AEL::              
* Using AEL::                   
@end menu

@node Installing AEL, Using AEL, User's Guide, User's Guide
@section Installing the Ada Ephemeris Library

@c ......................................................................
@menu
* Installing Ada files::        
* Retrieving the ephemeris files::  
* Testing::                     
@end menu

@node Installing Ada files, Retrieving the ephemeris files, Installing AEL, Installing AEL
@subsection Downloading and compiling the AEL distribution

The AEL can be dowloaded from @uref{http://www.dit.upm.es/jpuente/AEL}
as a set of Ada source and other auxiliary files packed in a
@code{.tgz} (for @acronym{UNIX} or @acronym{GNU}/Linux systems) or
@code{.zip} (for Windows@registeredsymbol{} systems) archive.

An Ada 2005 compilation system must be installed in order to compile
the library. @acronym{GNAT GPL 2006} is assumed in the following, but
any other Ada compilation system can be used as well. In this case,
refer to the appropriate manuals for compilation instructions.

To compile the library on a @acronym{UNIX} or @acronym{GNU}/Linux system,
unzip the archive into any directory (@var{<build-dir>}) and then type:

@example
$ cd @var{<build-dir>}
$ make
@end example

Instructions for building the library on Windows@registeredsymbol{}
systems are included in appendix 




@c ......................................................................
@node Retrieving the ephemeris files, Testing, Installing Ada files, Installing AEL
@subsection Retrieving and installing the ephemeris files

@c ......................................................................
@node Testing,  , Retrieving the ephemeris files, Installing AEL
@subsection Testing the library

@c ----------------------------------------------------------------------
@node Using AEL,  , Installing AEL, User's Guide
@section Using the Ada Ephemeris Library

@c ......................................................................
@menu
* Main packages::               
* Ephemeris::                   
* Ephemeris.Data::              
* Compiling with AEL packages::  
@end menu

@node Main packages, Ephemeris, Using AEL, Using AEL
@subsection Main packages

@c ......................................................................
@node Ephemeris, Ephemeris.Data, Main packages, Using AEL
@subsection The package @code{Ephemeris}


@c ......................................................................
@node Ephemeris.Data, Compiling with AEL packages, Ephemeris, Using AEL
@subsection The package @code{Ephemeris.Data}


@c ......................................................................
@node Compiling with AEL packages,  , Ephemeris.Data, Using AEL
@subsection Compiling with @acronym{AEL} packages


@c ======================================================================
@node Detailed design, Ephemeris file formats, User's Guide, Top
@chapter Detailed design

@menu
* Interpolation coefficients::  
@end menu

@c ......................................................................
@node Interpolation coefficients,  , Detailed design, Detailed design
@section Interpolation coefficients

Since computing the values of the state components (position and
velocity) for every point in time is not possible, the ephemeris
files provide instead a set of interpolation coefficients that can be used
to calculate the state at any given point of time. To this purpose,
time is divided into continous adjacent intervals of fixed length
called @emph{granules}. The state components as functions of time are
then expressed as a sum of Chebyshev polynomials.
  
@iftex
The expression of the position and velocity components of a body in
terms of Chebyshev polinomials is:
@end iftex

@tex
$$\eqalignno{
  p_i(\tau) &\doteq  \sum_{n=0}^{N}{a_n T_n(\tau)} & (1)\cr
  v_i(\tau)  &= \dot{p}_i(\tau) 
    \doteq \sum_{n=0}^{N}{a_n \dot{T}_n(\tau)} & (2)\cr
}
$$
@end tex

@iftex
@noindent
The independent variable @math{t} is the normalized time (-1.0..+1.0)
in a granule span:
@end iftex

@tex
$$\tau = {t -t_{0} \over t_{1} - t_{0}} \eqno (3)$$
@end tex

@iftex
@noindent
where @math{t} is the time for which the calculation is to be made,
and @math{t_0} and @math{t-1} are, respectively, the beginning and end
times of the granule in which @math{t} lies.
@end iftex

The Chebyshev polynomial of 




The length of a granule is not the same for all solar system
bodies. @ref{tab:granule} shows the granule lenght and the number of
Chebyshev polynomials used for the development of the state of the
main solar system bodies.

@float Table,tab:granule
@caption{Granule length and number of polynomials for the main solar
system bodies.}
@multitable {11} {Mercury and others} {Granule (days)} {No. of polynomials}
@headitem No. @tab Name    @tab Granule (days) @tab No. of polynomials
@item     1   @tab Mercury @tab  8      @tab 13
@item     2   @tab Venus   @tab 16      @tab  9

@end multitable

@end float


@c ======================================================================
@node Ephemeris file formats, Building the library on Windows systems, Detailed design, Top
@appendix Ephemeris file formats


@c ======================================================================
@node  Building the library on Windows systems, Copying this manual, Ephemeris file formats, Top
@appendix Building the library on Windows systems


@c ======================================================================
@node Copying this manual, Index, Building the library on Windows systems, Top
@appendix Copying this manual

@menu
* GNU Free Documentation License:: License for copying this manual
@end menu

@include fdl.texi
@c ======================================================================
@node Index,  , Copying this manual, Top
@unnumbered Index

@printindex cp

@bye

@c  LocalWords:  Chebyshev cartesian barycenter
