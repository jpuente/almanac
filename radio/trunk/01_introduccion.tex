%-------------------------------------------------------------
% $Id$
% © Juan A. de la Puente, 2008
%-------------------------------------------------------------
%!TEX root =  apuntes.tex
%-------------------------------------------------------------
\chapter{Introducción a las radiocomunicaciones}

\section{Radiocomunicaciones}

Las radiocomunicaciones son las comunicaciones que se efectúan por medio de ondas de radio. La información que se transmite por radio puede ser de varios tipos, por ejemplo:
\index{radiocomunicaciones}
%
\begin{itemize}
\item \emph{Sonido}: voz o música.
\item \emph{Imágenes}: fijas, como fotografías,  mapas o documentos, o móviles, como en la televisión.
\item \emph{Datos}: información digital de todo tipo, como mensajes de correo electrónico, documentos en general o páginas de web.
\end{itemize}
%
El sonido y las imágenes se pueden transmitir directamente o en forma numérica, como ocurre en la telefonía móvil y en la televisión digital. En este caso las características del sonido o la imagen se representan mediante una serie de números de acuerdo con un código determinado.

\subsection{Las radiocomunicaciones en la mar}

Desde comienzos del siglo \lsc{XX} la radio ha sido el medio fundamental para las comunicaciones entre barcos y entre éstos y tierra. La forma básica de comunicación en la mar es la \emph{radiotelefonía}, es decir la comunicación mediante voz entre dos o más interlocutores. 
\index{radiotelefonía}
La radiotelefonía es un elemento fundamental del Sistema Mundial de Salvamento y Seguridad Marítima  (SMSSM),%
\footnote{En inglés \emph{Global Maritime Distress and Safety System} (GMDSS).}
y también se usa de forma generalizada para efectuar comunicaciones entre barcos y entre éstos y terminales telefónicos en tierra. Otras formas de comunicación, basadas en técnicas digitales, permiten recibir información meteorológica e intercambiar documentos de todo tipo (facsímiles, correo electrónico, etc.). El SMSSM también utiliza técnicas digitales, como la \emph{llamada selectiva digital} (LSD),%
\footnote{DCS (\emph{Digital Selective Calling}).}
para enviar alertas de socorro y seguridad de forma eficiente.
\index{SMSSM,LSD,DSC}

% ojo
Históricamente tuvo mucha importancia la \emph{radiotelegrafía}, hoy en desuso. Esta forma de comunicación consiste en la transmisión de textos escritos mediante un código de impulsos largos (rayas) y cortos (puntos). El código radiotelegráfico más conocido es el código Morse,%
\footnote{Véase el apéndice \ref{ap:morse}.}
que debe su nombre a Samuel Morse, inventor del telégrafo. En España y otros muchos países dejó de usarse oficialmente la radiotelegrafía en las comunicaciones marítimas en 1999, como consecuencia de la implantación del SMSSM.
\index{radiotelegrafía}

\subsection{Servicios de radio}

Servicio fijo, servicio móvil
Servicio móvil marítimo

\section{Las ondas de radio}

Las ondas de radio son ondas electromagnéticas, es decir oscilaciones mantenidas de un campo electromagnético.

Forma básica: senoide.

Características:

\begin{itemize}
\item Frecuencia - Hz
\item Longitud metros - relación con la frecuencia
\item Polarización
\item Amplitud
\item Potencia
\end{itemize}

\subsection{Modulación}

variación de las propiedades de una onda

portadora, banda base

\subsection{Bandas de frecuencia}

asignación/atribución: ITU, DGCOM, DGMM

VHF, HF, MF / ondas métricas, decamétricas, hectométricas
%
\begin{table}[htdp]
\caption{Bandas de frecuencia}
\begin{center}
\begin{tabular}{cclccl}
%&\textit{Banda} & \textit{Frecuencia} & \textit{Longitud de onda} & \textit{Designación}\\ \hline

4& VLF & \textit{Very low frequency} &   3~-30 kHz  & 10 - 30~km & Ondas miriamétricas\\
5& LF   & \textit{Low frequency}         & 30~-300 kHz &  1 - 10 km   & Ondas kilométricas\\ 
6 &MF  & 300 - &3000 kHz & 100 - 1000 m & Ondas hectométricas\\
7 &HF  & 3 - 30 MHz & 10 - 100 m & Ondas decamétricas\\
8 &VHF & 30 - 300 MHz & 1 - 10 m  & Ondas métricas\\
\hline
\end{tabular}
\end{center}
\label{tb:bandas}
\end{table}%



\subsection{Propagación de las ondas}

onda directa y reflejada

Problemas : atenuación, ruido, interferencias

\section{Estaciones de radio}

Fija, móvil / 
Terrenal, espacial /
Terrestre
Estación de barco, estación costera

\section{Gestión de las radiocomunicaciones}

ITU / SOLAS / DGTEL / DGMM etc.

Reglamento de radiocomunicaciones - cuadro de atribución de bandas de frecuencias
