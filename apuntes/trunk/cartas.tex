%-------------------------------------------------------------
% $Id$
% © Juan A. de la Puente, 2008
%-------------------------------------------------------------
\chapter{Cartas náuticas}
\label{ch:cartas}
%===================================
\section{Cartas náuticas }

\index{cartas náuticas|textbf}

\index{carta|textbf}
\index{carta!náutica}
\index{carta!electrónica}
\index{visor}
\index{plotter|textbf}
\index{avisos a los navegantes}
\index{Instituto Hidrográfico de la Marina|textbf}

Una carta náutica es un mapa de una zona de la superficie del mar y de la costa adyacente.
Las cartas muestran la forma de la costa, la profundidad del mar, los puntos peligrosos para 
la navegación, las ayudas a la navegación, los puntos relevantes de la costa, y todos los 
detalles útiles para navegar en la zona que cubren. 

Las cartas pueden presentarse de distintas formas. La más común es la carta impresa 
sobre una superficie de papel, pero cada vez son más frecuentes las \emph{cartas electrónicas}, 
cuyo contenido está codificado en forma numérica y almacenado en un dispositivo electrónico adecuado. Las cartas electrónicas no son simplemente una versión digital de las cartas 
en papel, sino que pueden contener información adicional, que se puede consultar de distintas formas, y mezclar con los datos obtenidos de otros instrumentos electrónicos. Para 
leer las cartas electrónicas se necesita un visor adecuado, que puede ser un computador 
ordinario dotado de los programas necesarios. Sin embargo es muy frecuente que las cartas electrónicas 
se utilicen con un computador especializado (\emph{plotter}) integrado con otros instrumentos electrónicos, como 
navegadores por satélite o sistemas de radar. 

Los servicios hidrográficos de cada país realizan los trabajos necesarios para recopilar 
los datos necesarios para confeccionar cartas de las zonas de navegación que les interesan, 
y publican cartas oficiales basadas en esos datos y en los que intercambian con otros países. Estos mismos servicios o, en ocasiones, editoriales privadas publican también cartas 
especialmente adaptadas a la navegación deportiva. Es muy importante mantener las cartas 
al día, aplicándoles las correcciones que se publican en los avisos a los navegantes o en 
otras publicaciones similares. En España el Instituto Hidrográfico de la Marina es el organismo encargado de publicar las cartas náuticas oficiales y otros documentos destinados a los navegantes.

%------------------------------------------------------------


\subsection{La proyección de Mercator}

\index{proyección|textbf}
\index{proyección!de Mercator|textbf}
\index{proyección!equivalente}
\index{proyección!isogónica}
\index{proyección!conforme}
\index{línea!ortodrómica}
\index{línea!loxodrómica}
\index{carta!mercatoriana|textbf}



La representación de la superficie de la Tierra, aproximadamente esférica, sobre una superficie plana se realiza mediante una \emph{proyección}. Una proyección es una relación que hace 
corresponder a cada punto de la superficie terrestre un punto del plano de la carta.%
\footnote{En términos matemáticos este tipo de relación se llama \emph{aplicación}. Para construir un mapa de 
forma correcta la aplicación debe ser \emph{biyectiva}, es decir a cada punto de la parte de la superficie terrestre considerada debe corresponder un único punto en la carta y viceversa.}
 Generalmente se procura que la proyección utilizada tenga determinadas características, como la 
conservación de las formas de los accidentes geográficos (proyecciones conformes), de las superficies de los mares y continentes (proyecciones equivalentes), etc. Ninguna proyección 
permite mantener todas estas propiedades a la vez, por lo que generalmente es necesario 
sacrificar algún aspecto de la realidad para representar adecuadamente otros. En el caso 
de la navegación marítima es fundamental que los ángulos medidos en la carta sean iguales que los medidos en la realidad, ya que, como veremos, muchas técnicas de navegación 
se basan en la medida de los ángulos que forman determinadas líneas. 
Una proyección que tiene esta propiedad es una \emph{proyección conforme}. 

\begin{figure}[hbtp]
\begin{center}
\includegraphics[scale=0.7]{mercator}\\
\caption{Mapamundi en proyección de Mercator.}
\label{fg:mercator}
\end{center}
\end{figure}

La proyección de Mercator, ideada en 1569 por el cartógrafo flamenco Gerhard Kremer,%
\footnote{El nombre «Kremer » significa «mercader». «Mercator » es una versión latinizada de este 
nombre.}
 es la que se utiliza para confeccionar la gran mayoría de las cartas náuticas. Es una 
proyección conforme, en la que los meridianos se representan como líneas verticales, y 
los paralelos como líneas horizontales (figura\ref{fg:mercator}). Todos los demás círculos máximos 
(líneas ortodrómicas) se representan mediante curvas. Por el contrario, las líneas que 
siguen un rumbo constante (loxodrómicas) se representan como líneas rectas. Como 
hemos visto, esta es la propiedad más importante de esta proyección desde el punto de 
vista de la navegación. Sin embargo, la proyección de Mercator tiene el inconveniente de 
que la superficie de los continentes e islas resulta alterada, de tal manera que las 
tierras situadas en latitudes altas aparecen aumentadas con respecto a las situadas cerca del 
Ecuador. Así, Groenlandia aparece en las cartas mercatorianas con un tamaño similar al de 
África, cuando en realidad ésta es mucho mayor. 

%-------------------------------------------------------------------------------
\subsection{Escala de las cartas }

\index{carta!escala|textbf}
\index{escala|textbf}
\index{escala!\textemdash|seealso{carta}}

La escala de una carta es la relación entre el tamaño de los objetos representados en la 
carta y el tamaño real de los mismos en la superficie de la Tierra. Por ejemplo, si la escala 
es $1/100\,000$, una línea de 1~cm en la carta representa una distancia real de 1~km 
($=100\,000\,\mbox{cm}$). Por tanto, cuanto mayor es la escala de una carta, con mayor detalle se 
representará en ella la superficie terrestre.%
\footnote{Puesto que la escala se representa mediante una fracción cuyo numerador es igual a la unidad, 
será tanto mayor cuanto menor sea el valor del denominador. Así, una escala de $1/50\,000$ es 
mayor que una de $1/100\,000$. }

\begin{figure}[hbtp]
\begin{center}
\includegraphics[scale=0.45]{convergencia}\\
\caption{Convergencia de los meridianos.}
\label{fg:convergencia}
\end{center}
\end{figure}

En las cartas mercatorianas  la escala  no es uniforme,  sino que aumenta con la latitud. La razón de ello estriba en la necesidad de que la proyección sea conforme. Mientras que un arco de meridiano de 1'  mide lo mismo en todas las latitudes (1milla náutica),%
\footnote{Salvo una pequeña diferencia debida a la falta de esfericidad de la Tierra.}
un arco de paralelo de 1' sólo sólo mide 1 milla en el Ecuador. A medida que se avanza hacia latitudes más 
altas, los meridianos se van acercando (figura~\ref{fg:convergencia}), por lo que la longitud de los paralelos 
va disminuyendo con la latitud, de tal manera que en la latitud $\phi$ la longitud de un arco de 
1’ de paralelo es igual a $\cos \phi$ millas. Como en la proyección de Mercator la separación entre las líneas rectas que representan los meridianos es constante e independiente de la latitud,  es necesario aumentar la escala en el sentido de la latitud para mantener la proporción entre la medida de los arcos de meridiano y de paralelo en todas las latitudes.
Ésta es la razón por la cual las tierras situadas cerca de los Polos parecen mayores que las que están cerca del Ecuador. 

La figura \ref{fg:escalas-lat-long} muestra la representación en una carta mercatoriana de un rectángulo de 
1’ de latitud por 1’ de longitud. La relación entre la medida en la carta de los lados del rectángulo es $y = x/\cos \phi$,
siendo $\phi$ la latitud del lado inferior. 
Para diferentes latitudes tenemos:
\[ 
\begin{array}{rr}
\multicolumn{1}{c}{\phi} & y \\ \hline
0º & x \\
30º & 1.15 x\\
45º & 1.41 x \\
60º & 2.00 x\\
\end{array}
\]
Por tanto, cuanto mayor sea la latitud tanto mayor será la relación entre el lado vertical y el horizontal del rectángulo. Como la longitud del lado vertical es siempre la misma, 1~milla, vemos que la escala aumenta con la latitud.

\begin{figure}[hbtp]
\begin{center}
\includegraphics[scale=0.50]{escalas-lat-long}\\
\caption{Escalas de latitud y longitud.}
\label{fg:escalas-lat-long}
\end{center}
\end{figure}

%------------------------------------------------------------
\subsection{Otras proyecciones}

\index{proyección!gnomónica}
\index{proyección!gnomónica polar}
\index{proyección!polar estereográfica}
\index{proyección!de Lambert}
\index{derrota!ortodrómica}

La proyección de Mercator no es adecuada para representar las zonas polares, ya que por 
encima de los 60º de latitud la distorsión introducida por el aumento de la escala es muy 
grande, y los Polos no tienen representación. Por este motivo en las cartas de estas zonas 
se utilizan otras proyecciones, de las que la más común es la \emph{proyección gnomónica}. Ésta 
consiste en proyectar geométricamente la superficie terrestre desde el centro de la Tierra 
sobre un plano tangente a la misma. Cuando el punto de tangencia coincide con uno de los 
polos terrestres se denomina \emph{proyección gnomónica polar}. En este caso los meridianos se 
representan como rectas convergentes en el Polo, y los paralelos como circunferencias 
concéntricas con centro en el Polo. Otras proyecciones de uso frecuente en las cartas de las 
zonas polares son la \emph{proyección polar estereográfica} y la 
\emph{proyección de Lambert modificada}. 

La proyección gnomónica tiene la propiedad de que todos los círculos máximos se 
representan en las cartas como líneas rectas. Por este motivo a veces se usan cartas en pro- 
yección gnomónica oblicua (es decir, con el punto de tangencia en un lugar cualquiera de 
la superficie terrestre) para trazar derrotas ortodrómicas, es decir las que siguen un arco 
de círculo máximo. 

%===================================
\section{Características de las cartas}

\index{cartas náuticas!características}

Las cartas náuticas contienen numerosos detalles interesantes para la navegación, por lo 
que a veces su lectura puede resultar complicada a primera vista. Es necesario, por tanto, 
conocer los convenios de signos, rotulación, escalas, etc. que se utilizan en su confección, 
así como los procedimientos adecuados para mantenerlas al día. A continuación se resumen las características más importantes de las cartas oficiales españolas publicadas por el 
IHM (Instituto Hidrográfico de la Marina), que son similares a las de otros organismos. 
Características generales  

%--------------------------------------------------------------
\subsection{Identificación }

\index{cartas náuticas!identificación}

Las cartas náuticas se identifican mediante un número, que aparece en la esquina inferior 
derecha y en la superior izquierda de cada carta. Cerca del número situado en la esquina 
inferior derecha aparece también el número y la fecha de la edición, y la fecha de la última 
corrección efectuada en el momento de adquirir la carta (figura \ref{fg:id-carta}). 

\begin{figure}[htbp]
\begin{center}
\includegraphics[width=\textwidth]{numero-carta}\\
\caption{Datos de identificación de la carta número 47 del IHM.}
\label{fg:id-carta}
\end{center}
\end{figure}

%--------------------------------------------------------------
\subsection{Tarjeta}
 
\index{cartas náuticas!características}

La mayoría de la información general sobre la carta se encuentra en la \emph{tarjeta}, que es la 
parte de la carta donde se describe la zona que abarca la carta, junto con otros datos de 
interés (figura \ref{fg:tarjeta}). 

\begin{figure}[hbtp]
\begin{center}
\includegraphics[width=\textwidth]{tarjeta}\\
\caption{Tarjeta de la carta número 47 del IHM.}
\label{fg:tarjeta}
\end{center}
\end{figure}

%--------------------------------------------------------------
\subsection{Escala}

\index{carta!escala}
\index{escala}

La escala de la carta es, como hemos visto, la relación entre el tamaño de los objetos 
representados en la carta y su tamaño real sobre la superficie terrestre. Como en la proyección de Mercator la escala varía con la latitud, la escala que figura en la tarjeta se 
refiere siempre a una latitud determinada. Las cartas tienen también escalas gráficas de 
latitudes y longitudes en sus bordes, que permiten leer las coordenadas de un punto cualquiera y medir distancias, tal como se explica más adelante. Por ejemplo, la escala de la 
carta nº 47 del IHM es de 1:350 000 en la latitud 38º 30’ N (figura \ref{fg:tarjeta}). En esta misma 
carta, la escala es ligeramente mayor en latitudes más altas, y ligeramente menor en latitudes más bajas. 

\subsubsection{Escalas de latitudes y longitudes }

\index{carta!escala gráfica}
\index{escala!gráfica}

Las cartas náuticas llevan dos tipos de escalas gráficas, que se utilizan para medir coordenadas y distancias (figura \ref{fg:escala-grafica}): 
\begin{itemize}
\item La \emph{escala de latitudes} está situada en los bordes laterales de la carta. Se usa para 
medir la latitud de un punto en la carta. También se usa para medir distancias, ya 
que 1~M = 1’ de latitud. 
\item \emph{La escala de longitudes} está situada en los bordes superior e inferior. Se usa para 
medir longitudes sobre la carta. 
\end{itemize}

\begin{figure}[hbtp]
\begin{center}
\includegraphics[width=\textwidth]{escala-grafica}\\
\caption{Escalas de latitud y longitud.}
\label{fg:escala-grafica}
\end{center}
\end{figure}

\begin{ejemplo}
El punto marcado en la carta de la figura  \ref{fg:escala-grafica} está situado en 37º 41,3’ N  0º 45,6’ W.
\end{ejemplo}

%--------------------------------------------------------------
\subsection{Tipos de cartas}

\index{cartas náuticas!tipos}
\index{cartas!generales}
\index{cartas!de arrumbamiento|seealso{cartas de recalada}}
\index{cartas!de recalada}
\index{cartas!de navegación costera}
\index{aproche}
\index{portulano}
\index{carta!cartucho}

Las cartas se clasifican según su escala y la superficie de la Tierra que representan en: 
\begin{itemize}
\item \emph{Cartas generales}: las que representan una gran extensión de la superficie de la 
Tierra, con escalas comprendidas entre 1:30\,000\,000 y 1:3\,000\,000.Se utilizan 
únicamente para trazar grandes derrotas en navegación oceánica. 
\item \emph{Cartas de arrumbamiento} o \emph{recalada}: representan una extensión limitada de la 
superficie terrestre, con una escala comprendida entre 1:3\,000\,000 y 1:200\,000. 
Se utilizan para navegar en distancias medias o para acercarse a la costa después 
de una navegación de altura. 
\item \emph{Cartas para navegación costera}: representan de forma detallada una porción de 
costa, con una escala comprendida entre 1:200\,000 y 1: 50\,000. 
\item \emph{Aproches}: son cartas de gran escala (del orden de 1:25\,000), que se utilizan para 
navegar cerca de los puertos o en zonas de gran dificultad. 
\item  \emph{Portulanos}: con escala mayor de 1:25\,000. Representan con gran detalle el interior 
de los puertos, canales y fondeaderos. 
\end{itemize}
Los aproches y portulanos se encuentran a menudo insertados en forma de 
\emph{cartucho} en otras cartas de menor escala. 

\index{cartas!de punto menor}
\index{cartas!de punto mayor}

También se habla de cartas de \emph{punto menor} o de \emph{punto mayor}. Se consideran cartas de 
punto menor las cartas generales y las de arrumbamiento, y cartas de punto mayor las restantes. 

%--------------------------------------------------------------
\subsection{Datum de la carta} 

\index{datum}
\index{datum!horizontal}

\index{datum!europeo (Potsdam)}
%\index{sistema geodésico mundial|see{WGS84}}
%\index{World Geodetic System|see{WGS84}}
\index{coordenadas!geográficas}
 
La tarjeta también contiene información sobre el \emph{datum horizontal} de la carta. Como 
vimos en el capítulo \ref{ch:introduccion}, este término se refiere al sistema de referencia que 
que se utiliza para determinar las coordenadas geográficas sobre la superficie de la Tierra. 
Actualmente las cartas españolas se refieren al \emph{sistema geodésico mundial} (WGS\,84), 
que es el que se utiliza en la mayoría de los países y 
en los sistemas de navegación por satélite  (ver capítulo \ref{ch:satelite}). Sin embargo, todavía se encuentran cartas referidas al  \emph{datum europeo de 1950}. Con el fin de permitir el uso de estas cartas en la navegación por satélite, la tarjeta proporciona instrucciones para convertir las coordenadas geográficas medidas en la carta al sistema WGS84 y 
viceversa. 

%\begin{figure}[hbtp]
%\begin{center}
%\includegraphics[width=\textwidth]{datum-europeo}\\
%\caption{Tarjeta de una carta referida al datum europeo.}
%\label{fg:datum-europeo}
%\end{center}
%\end{figure}

\begin{ejemplo}
En la tarjeta de de una carta referida al datum europeo %(figura\ref{fg:datum-europeo})
figura el siguiente párrafo: 
\begin{quotation}\noindent\itshape
SITUACIONES OBTENIDAS POR SATÉLITE: Las situaciones obtenidas mediante sistemas 
de navegación por satélite referidas al Sistema Geodésico Mundial (WGS-84) deberán ser 
desplazadas 0,07 minutos al Norte y 0,07 minutos al Este para estar correctamente representadas en esta carta. 
\end{quotation}
Si se ha obtenido una situación por satélite referida al sistema WGS84 igual a 37º44,02’\,N
0º42,35’\,W, la trazaremos en la carta como 37º44,09’\,N 0º42,28’\,W. En una carta de punto mayor
la diferencia es imperceptible (0,4\,mm, aproximadamente, para una escala de 1:300\,000), pero si usamos cartas de 
punto mayor es muy importante hacer la correción del datum. Así, por ejemplo, la corrección de 0,07’ 
de latitud (129,6\,m) equivale a 2,6\,mm en una carta de escala 1:50\,000, y a 8,6\,mm en un portulano de escala
 1:15\,000.
\end{ejemplo}

Como vemos en el ejemplo anterior, cuando se quiere obtener situaciones precisas con 
sistemas de navegación por satélite es necesario tener en cuenta el datum de la carta y 
hacer las correcciones necesarias. 

\subsubsection{Datum vertical }

\index{datum!vertical}
\index{bajamar escorada}

El \emph{datum vertical} es la referencia que se usa para medir las elevaciones (en tierra) y las 
sondas (en la mar). En las cartas españolas las elevaciones se miden siempre en metros 
sobre el nivel medio del mar, y las sondas también en metros, referidas a la bajamar 
escorada (ver capítulo \ref{ch:mareas}). 

En las cartas inglesas y norteamericanas antiguas se miden las elevaciones en pies (ft),%
%\footnote{\emph{Feet}.} 
 y las sondas en brazas (fm)%
%\footnote{\emph{Fathoms}.}
 y pies, aunque en las cartas modernas estas magnitudes se 
miden en metros. En estas cartas las sondas se refieren al nivel medio de la bajamar de 
mareas vivas. 

%--------------------------------------------------------------
\subsection{Rosa náutica y declinación magnética}

\index{rosa}
\index{rosa!náutica}
\index{rosa!de los vientos|see rosa náutica}
\index{declinación!magnética}
\index{declinación!variación anual}

Una rosa náutica es una figura consistente en dos círculos graduados concéntricos que indican 
rumbos verdaderos y magnéticos, respectivamente (figura \ref{fg:rosa}). 
En la dirección que señala el Norte magnético se indica el valor de la declinación magnética 
para un año determinado, y el valor de la variación anual de la misma. 
Cuando la declinación magnética varía significativamente de una parte a otra de la 
carta, su valor se indica mediante distintas rosas o con rótulos adicionales. En estos casos 
hay que emplear el valor de la declinación indicado en la rosa o rótulo más próximo a la 
zona de la carta con la que se esté trabajando. 

\begin{figure}[htbp]
\begin{center}
\includegraphics[scale=0.5]{rosa}\\
\caption{Rosa náutica.}
\label{fg:rosa}
\end{center}
\end{figure}

\begin{ejemplo}
El valor de la declinación magnética en la zona de la carta donde se encuentra la rosa 
de la figura \ref{fg:rosa} es de 1º10’\,W en 2002. La variación anual es de 7’\,E. 
Por tanto, en 2009 la declinación será:
\[
\begin{array}{lcll}
  \mbox{Valor en 2002} & =  & -1º \, 10' \: \mathrm{(W)} \\
  \mbox{Variación}     & =  & +0º \, 42' \: \mathrm{(E)} & (\mbox{7 años} \times 10' )\\
%  \cline{3-5}\\
\mbox{Valor en 2009}  &  = &   -0º \, 25’ \: \mathrm{(W)} & \approx -0,5º 
\end{array}
\]
El valor de la declinación en 2003, por tanto, es aproximadamente $-0,5º$. 
\end{ejemplo}

%%%%%%%%%%%%%%%%%%%%%%%%

\subsubsection{Detalles de las cartas}

Las figuras 2.9 y 2.10 muestran sendos fragmentos de dos cartas náuticas: el aproche a la 
ría de Cedeira incluido en la carta nº930, y la carta de navegación costera nº471A del 
IHM. En ambas se pueden apreciar la mayoría de los detalles que suelen aparecer en las 
cartas náuticas. 

Signos y abreviaturas 

La publicación especial nº 14 (INT 1) del Instituto Hidrográfico de la Marina contiene 
todos los símbolos, abreviaturas y términos usados en las cartas oficiales españolas. Los 
símbolos son los recomendados por la Organización Hidrográfica Internacional (OHI), 
aunque en algunos casos las cartas españolas de una cierta antigüedad utilizan símbolos 
nacionales distintos de aquéllos. Estos símbolos nacionales se explican también en la 
publicación nº 14. 


Profundidad y naturaleza del fondo 
La profundidad medida en distintos puntos 
de la carta se indica mediante sondas. Las 
sondas se refieren al datum vertical de la 
carta (la bajamar escorada en las cartas espa- 
ñolas), y se representan mediante números 
que indican la profundidad en metros y, en su 
caso, en decímetros, en el punto donde se 
encuentran. La cifra que indica los decíme- 
tros se escribe como un subíndice de las que 
indican los metros enteros. Las sondas que se 
encuentran fuera de posición, es decir las que 
se refieren a un punto cercano a aquél donde 
se encuentran, se escriben entre paréntesis. 
La sondas medidas con precisión se escriben 
en cursiva, mientras que las dudosas o las 
obtenidas de otros documentos menos fiables 
se escriben con cifras rectas (figura 2.8). 
Las sondas negativas corresponden a 
puntos que velan en la bajamar escorada, aunque pueden quedar sumergidos al subir la 
marea (capítulo 8). Se escriben subrayadas. 

Ejemplo 2.5. En un punto en donde la carta marca 18 la profundidad del agua es igual a 0,7m cuando 
la altura de la marea es de 2,5m. Cuando la altura de la marea es menor de 1,8m el punto vela (es 
decir, está fuera del agua). 


Las sondas suelen ir acompañadas de una
abreviatura, situada debajo de ellas, que
indica la naturaleza del fondo. La figura 2.11
muestra algunas de las abreviaturas más
comunes definidas por las normas nacionales
e internacionales. 
Los veriles o líneas isobáticas son líneas
que unen todos los puntos que tienen una
misma profundidad (por ejemplo, 10m, 20m
o 100m). Las zonas delimitadas por los veri-
les de menor profundidad, dependiendo de la
escala y del tipo de carta, se colorean con dis-
tintos tonos de azul. La zona cercana a la
costa que queda al descubierto en la bajamar
se colorea de verde claro. 
 Peligros 
Los distintos peligros que se encuentran en la
mar, como piedras, bajos, restos de naufra-
gios, etc. se indican en las cartas mediante
diversos signos, los más importantes de los
cuales se muestran en la figura 2.12. 
Marcas y luces 
Las cartas muestran de forma detallada la
situación de las marcas terrestres útiles para la
navegación, y de las boyas, balizas, luces y
otras ayudas. Las luces se indican con una
marca púrpura, junto con sus características
luminosas (figura 2.13). Los sectores visibles
o de distintos colores, en su caso, se indican
también gráficamente en la carta. También se
indica la altura de la luz sobre el nivel medio
del mar y su alcance en millas. Todos estos
datos se encuentran igualmente en el Libro de 
Faros, donde debe acudirse siempre para una 
referencia más completa y exacta. 
V, G Verde (green). 
Figura 2.13. Símbolos y abreviaturas 
de luces y faros.

Puesta al día de las cartas 
La información que contienen las cartas sólo es exacta en la fecha de su última actualiza- 
ción, que en el momento de su adquisición está escrita en el margen inferior (figura 2.4). 
Para mantenerlas al día es preciso actualizarlas con las correcciones que se publican cada 
semana en los Avisos a los navegantes, editados por el Instituto Hidrográfico de la 
Marina. 

%===================================
\section{Trabajo sobre la carta}


%===================================
\section{Cartas electrónicas}
